% В этом файле следует писать текст работы, разбивая его на
% разделы (section), подразделы (subsection) и, если нужно,
% главы (chapter).

% Предварительно следует указать необходимую информацию
% в файле SETUP.tex

%% В этот файл не предполагается вносить изменения

% В этом файле следует указать информацию о себе
% и выполняемой работе.

\documentclass [fontsize=14pt, paper=a4, pagesize, DIV=calc]%
{scrartcl}
% ВНИМАНИЕ! Для использования глав поменять
% scrartcl на scrreprt

% Здесь ничего не менять
\usepackage [T2A] {fontenc}   % Кириллица в PDF файле
\usepackage [utf8] {inputenc} % Кодировка текста: utf-8
\usepackage [russian] {babel} % Переносы, лигатуры

%%%%%%%%%%%%%%%%%%%%%%%%%%%%%%%%%%%%%%%%%%%%%%%%%%%%%%%%%%%%%%%%%%%%%%%%
% Создание макроса управления элементами, специфичными
% для вида работы (курс., бак., маг.)
% Здесь ничего не менять:
\usepackage{ifthen}
\newcounter{worktype}
\newcommand{\typeOfWork}[1]
{
	\setcounter{worktype}{#1}
}

% ВНИМАНИЕ!
% Укажите тип работы: 0 - курсовая, 1 - бак., 2 - маг.,
% 3 - бакалаврская с главами.
\typeOfWork{0}
% Считается, что курсовая и бак. бьются на разделы (section) и
% подразделы (subsection), а маг. — на главы (chapter), разделы и
%  подразделы. Если хочется,
% чтобы бак. была с главами (например, если она большая),
% надо выбрать опцию 3.

% Если при выборе 2 или 3 вы забудете поменять класс
% документа на scrreprt (см. выше, в самом начале),
% то получите ошибку:
% ./aux/appearance.tex:52: Package scrbase Error: unknown option ` chapterprefix=

%%%%%%%%%%%%%%%%%%%%%%%%%%%%%%%%%%%%%%%%%%%%%%%%%%%%%%%%%%%%%%%%%%%%%%%%
% Информация об авторе и работе для титульной страницы

\usepackage {titling}

% Имя автора в именительном падеже (для маг.)
\newcommand {\me}{%
А.\,С.~Болотина%
}

% Имя автора в родительном падеже (для курсовой и бак.)
\newcommand {\byme}{%
А.\,С.~Болотиной%
}

% Научный руководитель
\newcommand{\supervisor}%
{асс.~каф.~ИВЭ~А.\,М.~Пеленицын}

% идентифицируем пол (только для курсовой и бак.)
\newcommand{\bystudent}{
Студентки
}

% Год публикации
\date{2017}

% Название работы
\title{Обобщённое программирование в задаче\\построения зипперов}

% Кафедра
%

\newcommand {\direction} {%
Направление подготовки\\02.\ifthenelse{\value{worktype} = 2}{04}{03}.02 ---
Фундаментальная информатика\\и информационные технологии%
}

%%%%%%%%%%%%%%%%%%%%%%%%%%%%%%%%%%%%%%%%%%%%%%%%%%%%%%%%%%%%%%%%%%%%%%%%
% Другие настраиваемые элементы текста

% Листинги с исходным кодом программ: укажите язык программирования
\usepackage{listings}
\lstset{
    language=Haskell,
    basicstyle=\small\ttfamily,
    breaklines=true,%
    showstringspaces=false,%
	%frame=single
    %inputencoding=utf8x%
}
% полный список языков, поддерживаемых данным пакетом, есть,
% например, здесь (стр. 13):
% ftp://ftp.tex.ac.uk/tex-archive/macros/latex/contrib/listings/listings.pdf

% Нумерация списков: можно при необходимести
% изменять вид нумерации (например, добавлять правую скобку).
% По умолчанию буду списки вида:
% 1.
% 2.
% Изменять вид нумерации можно в начале нумерации:
% \begin{enumerate}[1)] (В квадратных скобках указан желаемый вид)
\usepackage[shortlabels]{enumitem}
                    \setlist[enumerate, 1]{1.}

% Гиперссылки: настройте внешний вид ссылок
\usepackage%
[pdftex,unicode,pdfborder={0 0 0},draft=false,%backref=page,
    hidelinks, % убрать, если хочется видеть ссылки: это
               % удобно в PDF файле, но не должно появиться на печати
    bookmarks=true,bookmarksnumbered=false,bookmarksopen=false]%
{hyperref}


\usepackage {amsmath}      % Больше математики
\usepackage {amssymb}
\usepackage {textcase}     % Преобразование к верхнему регистру
\usepackage {indentfirst}  % Красная строка первого абзаца в разделе

\usepackage {fancyvrb}     % Листинги: определяем своё окружение Verb
\DefineVerbatimEnvironment% с уменьшенным шрифтом
	{Verb}{Verbatim}
	{fontsize=\small}

\DeclareMathAlphabet{\m}{T1}{PTMono-TLF}{m}{n}
\DeclareMathAlphabet{\mb}{T1}{PTMono-TLF}{b}{n}
\DeclareMathAlphabet{\mi}{T1}{PTMono-TLF}{m}{it}

\newcommand{\ind}{$\hphantom{~~}$}
\newcommand{\texttts}[1]{{\ttfamily\small #1}}
\newcommand{\tms}{\hspace{-0,45mm}$\times$\hspace{-0,45mm}}

% Вставка рисунков
\usepackage {graphicx}

% Общее оформление
% ----------------------------------------------------------------
% Настройка внешнего вида

%%% Шрифты

% если закомментировать всё — консервативная гарнитура Computer Modern
\usepackage{paratype} % профессиональные свободные шрифты
%\usepackage {droid}  % неплохие свободные шрифты от Google
%\usepackage{mathptmx}
%\usepackage {mmasym}
%\usepackage {psfonts}
%\usepackage{lmodern}
%var1: lh additions for bold concrete fonts
%\usepackage{lh-t2axccr}
%var2: the package below could be covered with fd-files
%\usepackage{lh-t2accr}
%\usepackage {pscyr}

% Геометрия текста

\usepackage{setspace}       % Межстрочный интервал
\onehalfspacing

\newlength\MyIndent
\setlength\MyIndent{1.25cm}
\setlength{\parindent}{\MyIndent} % Абзацный отступ
\frenchspacing            % Отключение лишних отступов после точек
\KOMAoptions{%
    DIV=calc,         % Пересчёт геометрии
    numbers=endperiod % точки после номеров разделов
}

                            % Консервативный вариант:
%\usepackage                % ручное задание геометрии
%[%                         % (не рекомендуется в проф. типографии)
%  margin = 2.5cm,
  %includefoot,
  %footskip = 1cm
%] %
%  {geometry}

%%% Заголовки


\ifthenelse{{\value{worktype} > 1}}{%
  \KOMAoptions{%
      headings=normal,   % размеры заголовков поменьше стандартных
      chapterprefix=true,% Печатать слово Глава
      appendixprefix=true% Печатать слово Приложение
  }
}{% Печатать слово Приложение даже если нет глав
  \newcommand*{\appendixmore}{%
    \renewcommand*{\sectionformat}{%
    \appendixname~\thesection\autodot\enskip}
    \renewcommand*{\sectionmarkformat}{%
      \appendixname~\thesection\autodot\enskip}
  }
}

% шрифт для оформления глав и названия содержания
\newcommand{\SuperFont}{\Large\sffamily\bfseries}

% Заголовок главы
\ifthenelse{\value{worktype} > 1}{%
\renewcommand{\SuperFont}{\Large\normalfont\sffamily}
\newcommand{\CentSuperFont}{\centering\SuperFont}
\usepackage{fncychap}
\ChNameVar{\SuperFont}
\ChNumVar{\CentSuperFont}
\ChTitleVar{\CentSuperFont}
\ChNameUpperCase
\ChTitleUpperCase
}

% Заголовок (под)раздела с абзацного отступа
\addtokomafont{sectioning}{\hspace{\MyIndent}}

%\renewcommand*{\captionformat}{~---~}
%\renewcommand*{\figureformat}{Рисунок~\thefigure}

\addtokomafont{captionlabel}{\bfseries}
\renewcommand*{\figureformat}{Листинг~\thefigure}
\usepackage[font=small,skip=3pt]{caption}%
\addtolength{\intextsep}{.15cm}
\setlength{\belowcaptionskip}{-.3cm}
\addtolength{\floatsep}{.8cm}
\addtolength{\textfloatsep}{.2cm}

% Плавающие листинги
\usepackage{float}
\floatstyle{ruled}
\usepackage{framed}
%\floatstyle{boxed}
%\restylefloat{figure}
\floatname{ListingEnv}{Листинг}
\newfloat{ListingEnv}{htbp}{lol}[section]

% точка после номера листинга
\makeatletter
\renewcommand\floatc@ruled[2]{{\@fs@cfont #1.} #2\par}
\makeatother


%%% Оглавление
\usepackage{tocloft}

% шрифт и положение заголовка
\ifthenelse{\value{worktype} > 1}{%
\renewcommand{\cfttoctitlefont}{\hfil\SuperFont\MakeUppercase}
}{
\renewcommand{\cfttoctitlefont}{\hfil\SuperFont}
}

% слово Глава
\usepackage{calc}
\ifthenelse{\value{worktype} > 1}{%
\renewcommand{\cftchappresnum}{Глава }
\addtolength{\cftchapnumwidth}{\widthof{Глава }}
}

% Очищаем оформление названий старших элементов в оглавлении
\ifthenelse{\value{worktype} > 1}{%
\renewcommand{\cftchapfont}{}
\renewcommand{\cftchappagefont}{}
}{
\renewcommand{\cftsecfont}{}
\renewcommand{\cftsecpagefont}{}
}

% Точки после верхних элементов оглавления
\renewcommand{\cftsecdotsep}{\cftdotsep}
%\newcommand{\cftchapdotsep}{\cftdotsep}

\ifthenelse{\value{worktype} > 1}{%
    \renewcommand{\cftchapaftersnum}{.}
}{}
\renewcommand{\cftsecaftersnum}{.}
\renewcommand{\cftsubsecaftersnum}{.}
\renewcommand{\cftsubsubsecaftersnum}{.}

%%% Списки (enumitem)

\usepackage {enumitem}      % Списки с настройкой отступов
\setlist %
{ %
  leftmargin = \parindent, itemsep=.5ex, topsep=.4ex
} %

% По ГОСТу нумерация должны быть буквами: а, б...
%\makeatletter
%    \AddEnumerateCounter{\asbuk}{\@asbuk}{м)}
%\makeatother
%\renewcommand{\labelenumi}{\asbuk{enumi})}
%\renewcommand{\labelenumii}{\arabic{enumii})}

%%% Таблицы: выбрать более подходящие

\usepackage{booktabs} % считаются наиболее профессионально выполненными
%\usepackage{ltablex}
%\newcolumntype {L} {>{---}l}

%%% Библиография

\usepackage{csquotes}        % Оформление списка литературы
\usepackage[
  backend=biber,
  hyperref=auto,
  sorting=none, % сортировка в порядке встречаемости ссылок
  language=auto,
  citestyle=gost-numeric,
  bibstyle=gost-numeric
]{biblatex}
\addbibresource{biblio.bib} % Файл с лит.источниками

% Настройка величины отступа в списке
\ifthenelse{\value{worktype} < 2}{%
\defbibenvironment{bibliography}
  {\list
     {\printtext[labelnumberwidth]{%
    \printfield{prefixnumber}%
    \printfield{labelnumber}}}
     {\setlength{\labelwidth}{\labelnumberwidth}%
      \setlength{\leftmargin}{\labelwidth}%
      \setlength{\labelsep}{\dimexpr\MyIndent-\labelwidth\relax}% <----- default is \biblabelsep
      \addtolength{\leftmargin}{\labelsep}%
      \setlength{\itemsep}{\bibitemsep}%
      \setlength{\parsep}{\bibparsep}}%
      \renewcommand*{\makelabel}[1]{\hss##1}}
  {\endlist}
  {\item}
}{}

% ----------------------------------------------------------------
% Настройка переносов и разрывов страниц

\binoppenalty = 10000      % Запрет переносов строк в формулах
\relpenalty = 10000        %

\sloppy                    % Не выходить за границы бокса
%\tolerance = 400          % или более точно
\clubpenalty = 10000       % Запрет разрывов страниц после первой
\widowpenalty = 10000      % и перед предпоследней строкой абзаца

% ----------------------------


% Стили для окружений типа Определение, Теорема...
% Оформление теорем (ntheorem)

\usepackage [thmmarks, amsmath] {ntheorem}
\theorempreskipamount 0.6cm

\theoremstyle {plain} %
\theoremheaderfont {\normalfont \bfseries} %
\theorembodyfont {\slshape} %
\theoremsymbol {\ensuremath {_\Box}} %
\theoremseparator {:} %
\newtheorem {mystatement} {Утверждение} [section] %
\newtheorem {mylemma} {Лемма} [section] %
\newtheorem {mycorollary} {Следствие} [section] %

\theoremstyle {nonumberplain} %
\theoremseparator {.} %
\theoremsymbol {\ensuremath {_\diamondsuit}} %
\newtheorem {mydefinition} {Определение} %

\theoremstyle {plain} %
\theoremheaderfont {\normalfont \bfseries} 
\theorembodyfont {\normalfont} 
%\theoremsymbol {\ensuremath {_\Box}} %
\theoremseparator {.} %
\newtheorem {mytask} {Задача} [section]%
\renewcommand{\themytask}{\arabic{mytask}}

\theoremheaderfont {\scshape} %
\theorembodyfont {\upshape} %
\theoremstyle {nonumberplain} %
\theoremseparator {} %
\theoremsymbol {\rule {1ex} {1ex}} %
\newtheorem {myproof} {Доказательство} %

\theorembodyfont {\upshape} %
%\theoremindent 0.5cm
\theoremstyle {nonumberbreak} \theoremseparator {\\} %
\theoremsymbol {\ensuremath {\ast}} %
\newtheorem {myexample} {Пример} %
\newtheorem {myexamples} {Примеры} %

\theoremheaderfont {\itshape} %
\theorembodyfont {\upshape} %
\theoremstyle {nonumberplain} %
\theoremseparator {:} %
\theoremsymbol {\ensuremath {_\triangle}} %
\newtheorem {myremark} {Замечание} %
\theoremstyle {nonumberbreak} %
\newtheorem {myremarks} {Замечания} %


% Титульный лист
% Макросы настройки титульной страницы
% В этот файл не предполагается вносить изменения

%\usepackage {showframe}

% Вертикальные отступы на титульной странице
\newcommand{\vgap}{\vspace{16pt}}

% Помещение города и даты в нижний колонтитул
\usepackage{scrlayer}
\DeclareNewLayer[
  foot,
  foreground,
  contents={%
    \raisebox{\dp\strutbox}[\layerheight][0pt]{%
      \parbox[b]{\layerwidth}{\centering Ростов-на-Дону\\ \thedate%
       \\\mbox{}
       }}%
  }
]{titlepage.foot.fg}
\DeclareNewPageStyleByLayers{titlepage}{titlepage.foot.fg}


\AtBeginDocument %
{ %
  %
  \begin{titlepage}
  %
    \thispagestyle{titlepage}

    {\centering
    %
    \MakeTextUppercase {МИНИСТЕРСТВО ОБРАЗОВАНИЯ И НАУКИ РФ}

    \vgap

    Федеральное государственное автономное образовательное\\
    учреждение высшего образования\\
    \MakeTextUppercase {Южный федеральный университет}

    \vgap

	Институт математики, механики и компьютерных наук
    имени~И.\,И.\,Воровича

    \vgap

    \direction

    \vspace* {\fill}

    \ifthenelse{\value{worktype} = 2}{%
    \me

    \vgap}{}

    {\usefont{T2A}{PTSansCaption-TLF}{m}{n}
    \MakeTextUppercase{\thetitle}}

    \ifthenelse{\value{worktype} = 2}{%
     \vgap

    Магистерская диссертация}{}
    \ifthenelse{\value{worktype} = 0}{
     \vgap

    Курсовая работа
    }{}%
    \ifthenelse{\value{worktype} = 1 \OR \value{worktype} = 3}{
     \vgap

    Выпускная квалификационная работа\\
    на степень бакалавра
    }{}%

    \vspace {\fill}

    \begin{flushright}
    \ifthenelse{\value{worktype} = 0 \OR 
                \value{worktype} = 1 \OR
                \value{worktype} = 3}{
      \bystudent \ifthenelse{\value{worktype} = 0}{3}{4}\ курса\\
      \byme
    }{}

    \vgap

    Научный руководитель:\\
    \supervisor\\
    \ifthenelse{\value{worktype} = 2}{%
    Рецензент:\\
    ученая степень, ученое звание, должность
    И. О. Фамилия
    }{}
	\end{flushright}
\ifthenelse{\value{worktype} = 0}{
\vspace{\fill}
        \begin{flushleft}
          \begin{tabular}{cc}
            \underline{\hspace{4cm}}&\underline{\hspace{5cm}}\\
            {\small оценка (рейтинг)} & {\small  подпись руководителя}\\
          \end{tabular}
          \\[1cm]
        \end{flushleft}
}{}
\ifthenelse{\value{worktype} = 1 \OR \value{worktype} = 3}{
\vspace{\fill}
        \begin{flushleft}
Допущено к защите:\\руководитель направления ФИИТ
\underline{\hspace{4cm}}
В.\,С.\,Пилиди
        \end{flushleft}
}{}


  	\vspace {\fill}
  %Ростов-на-Дону

    %\thedate

  }\end{titlepage}
  %
  %
  \tableofcontents
  %
  \clearpage
} %


% Команды для использования в тексте работы


% макросы для начала введения и заключения
\newcommand{\Intro}{\addsec{Введение}}
\ifthenelse{\value{worktype} > 1}{%
    \renewcommand{\Intro}{\addchap{Введение}}%
}

\newcommand{\Conc}{\addsec{Заключение}}
\ifthenelse{\value{worktype} > 1}{%
    \renewcommand{\Conc}{\addchap{Заключение}}%
}

% Правильные значки для нестрогих неравенств и пустого множества
\renewcommand {\le} {\leqslant}
\renewcommand {\ge} {\geqslant}
\renewcommand {\emptyset} {\varnothing}

% N ажурное: натуральные числа
\newcommand {\N} {\ensuremath{\mathbb N}}

% значок С++ — используйте команду \cpp
\newcommand{\cpp}{%
C\nolinebreak\hspace{-.05em}%
\raisebox{.2ex}{+}\nolinebreak\hspace{-.10em}%
\raisebox{.2ex}{+}%
}

% Неразрывный дефис, который допускает перенос внутри слов,
% типа жёлто-синий: нужно писать жёлто"/синий.
\makeatletter
    \defineshorthand[russian]{"/}{\mbox{-}\bbl@allowhyphens}
\makeatother


\endinput

% Конец файла


\begin{document}

\Intro

Идея обобщённого программирования (\textsl{generic programming})
состоит в том, чтобы увеличить гибкость языков программирования,
расширяя возможности для параметризации программ типами данных.
Термин \emph{обобщённое программирование} может иметь различное значение в зависимости от контекста. Говоря об
объектно-ориентированных языках, под ним понимают чаще всего
параметрический полиморфизм в языках, основанных на системе типов
Хиндли---Милнера, либо библиотеки обобщённых алгоритмов и
структур данных, метапрограммирование и так далее. Тем не менее,
когда говорят о функциональном программировании, его определение
относится к структурному полиморфизму~\cite{Loh2004}. Это
означает, что функции определяются над структурой типов данных. В
данной работе речь будет идти об \emph{обобщённом
программировании типов данных} (\textsl{datatype generic
programming})~\cite{Gib2007}, то есть о возможности перевода типа
в некоторое специальное \emph{представление}
(\textsl{representation}) его внутренней структуры и написания
функций, параметризованных такими представлениями и способных
единообразно их обрабатывать.

В работе изучается подход к обобщённому программированию типов
данных на языке Haskell, описанный в статье~\cite{VriLoh2014} и
реализованный в библиотеке \textsf{generics-sop}~\cite{generics-sop}.
Исследуется возможность применения средств подхода в задаче
построения зипперов~---~структур данных, используемых для
эффективной, чисто функциональной навигации по древовидной
структуре.

Раздел~\ref{sec:generic-prog} включает предварительные сведения об
обобщённом программировании типов данных. В
разделе~\ref{sec:zippers} содержится описание структуры данных
<<Зиппер>>, а также задачи, решаемой при помощи этой структуры,
излагается идея дифференцирования типов данных и раскрывается его
связь с типом зиппера, приводится набор правил дифференцирования,
который является результатом статьи~\cite{McBr2001}, позволяющий
механизировать получение зиппера, и рассматривается более ранний
известный подход к реализации обобщённого зиппера,
представленный в статье~\cite{MuRec2009}. Этот подход использует
средства библиотеки обобщённого программирования
\textsf{multirec}~\cite{multirec}, недостаток которой заключается в
необходимости переводить данный тип в его обобщённое
представление вручную или используя расширение языка
\textsf{Template Haskell}~\cite{multirec-th}, которое является
тяжеловесным средством метапрограммирования. В работе показано,
что библиотека \textsf{generics-sop} позволяет генерировать такое
представление автоматически, используя встроенные возможности
компилятора \textsf{GHC}.

В разделе~\ref{sec:generics-sop} представлен обзор возможностей
библиотеки \textsf{generics-sop}: описываются идеи и особенности
нового подхода к обобщённому программированию, приводится
несколько примеров описания обобщённых функций с помощью
\textsf{generics-sop}.

В результате исследования получен механизм, позволяющий
автоматизировать процесс построения обобщённого представления
зиппера средствами данной библиотеки, он описывается в
разделе~\ref{sec:generic-zippers-sop}.

\section{Введение в обобщённое программирование на языке Haskell}
\label{sec:generic-prog}

Множество примеров применения обобщённого программирования
типов данных включает большое количество функций в языке Haskell,
которые могут быть определены систематическим образом для
широкого класса типов: функции проверки на равенство и сравнения,
различные виды преобразований между значениями типов и другими
форматами представления данных (JSON, XML и т.~д.), функции
обхода или навигации по структурам данных, а также доступа к
конкретным данным внутри структуры (линзы) и другие.

Чтобы раскрыть идею обобщённого программирования, рассмотрим
следующий пример. Предположим, что у нас есть тип данных
\texttts{A}, опишем функцию проверки на равенство для этого типа
(листинг~\ref{list:eq-a}).
\begin{figure}[h]
\begin{framed}
\ttfamily\small
eq$_\mb A$ :: A -> A -> \textbf{Bool}
\end{framed}
\caption{Функция проверки на равенство для типа \texttts{A}}
\label{list:eq-a}
\end{figure}

Определение функции \texttts{eq$_\mb{A}$} несложно дать, имея
определение типа \texttts{A}. Алгоритм неформально можно описать
так: если тип данных имеет несколько конструкторов, необходимо
проверить, что для двух аргументов выбран один и тот же
конструктор, и если это так, то сравнить поэлементно на равенство
аргументы конструктора.

Например, так определяется функция \texttts{eq$_\mb{Bool}$} для
элементарного типа данных \lstinline{Bool}, который имеет два
конструктора без аргументов (листинг~\ref{list:eq-bool}).
\begin{figure}[h]
\begin{framed}
\ttfamily\small
\textbf{data} Bool = True | False\\
\\
eq$_\mb{Bool}$ :: \textbf{Bool} -> \textbf{Bool} -> \textbf{Bool}\\
eq$_\mb{Bool}$ True~ True~ = True\\
eq$_\mb{Bool}$ False False = True\\
eq$_\mb{Bool}$ \_~~~~ \_~~~~ = False
\end{framed}
\caption{Функция проверки на равенство для типа \lstinline{Bool}}
\label{list:eq-bool}
\end{figure}

Предположим теперь, что есть класс типов (листинг~\ref{list:generic}),
\begin{figure}[h]
\begin{framed}
\vspace{-0.25cm}
\begin{lstlisting}
class Generic a where
  type Rep a
  from :: a -> Rep a
  to   :: Rep a -> a
\end{lstlisting}
\vspace{-0.25cm}
\end{framed}
\caption{Класс типов, представимых в обобщённом виде}
\label{list:generic}
\end{figure}
который связывает тип \texttts{a}, любой экземпляр этого класса, c
изоморфным ему типом обобщённого представления \texttts{Rep a},
определяя преобразование между ними с помощью функций
\texttts{from} и \texttts{to}.

Теперь, если все типы \texttts{Rep a} имеют общую структуру, мы
можем ввести класс типов, определяющий функцию сравнения на
равенство \texttts{geq}, которая работает для всех типов
представления по индукции над их структурой, и определить с её
помощью функцию \texttts{eq} для любых типов, представимых в
обобщённом виде (листинг~\ref{list:geq}).

\begin{figure}[h]
\begin{framed}
\vspace{-0.25cm}
\begin{lstlisting}
class GEq repA where
  geq :: repA -> repA -> Bool

eq :: (Generic a, GEq (Rep a)) => a -> a -> Bool
eq x y = geq (from x) (from y)
\end{lstlisting}
\vspace{-0.25cm}
\end{framed}
\caption{Определение обобщённой функции сравнения на равенство}
\label{list:geq}
\end{figure}

\subsection{Алгебраические типы данных}

Алгебраические типы данных
(АТД, \textsl{algebraic data types})~---~основной механизм,
использованный для реализации структур данных в функциональных
языках программирования. Они определяются как составные типы,
которые могут быть представлены в виде типов"/сумм из
типов"/произведений. Тип"/произведение соответствует декартову
произведению множеств значений типов, а тип"/сумма в теории
множеств совпадает с дизъюнктивным, или размеченным,
объединением, то есть множеством, элементами которого являются
пары, состоящие из метки (соответствующей конструктору) и
сопоставляемого с ней типа"/произведения (представляющего
аргументы конструктора).

Приведём наиболее простые примеры алгебраических типов данных.
Базовыми случаями являются единичный тип, то есть тип, состоящий
из одного конструктора без аргументов, нулевой тип~---~тип, не
имеющий конструкторов, и тип"/константа, единственный конструктор
которого принимает один аргумент. Примеры таких типов
представлены в листинге~\ref{list:base}. Слева от знака \texttts{=}
стоят конструкторы типов, а справа~---~конструкторы значений.
Примеры элементарных типа"/суммы и типа"/произведения приведены
в листингах~\ref{list:sum}--\ref{list:prod}.
\begin{figure}[h]
\begin{framed}
\vspace{-0.25cm}
\begin{lstlisting}
data Unit = Unit

data Zero

data Const a = Const a
\end{lstlisting}
\vspace{-0.25cm}
\end{framed}
\caption{Базовые АТД: единичный тип, нулевой тип и тип"/константа}
\label{list:base}
\end{figure}
\begin{figure}[h]
\begin{framed}
{\ttfamily\small
\textbf{data} Either a b = Left a | Right b}
\end{framed}
\caption{Тип-сумма}
\label{list:sum}
\end{figure}
\begin{figure}[h]
\begin{framed}
{\ttfamily\small
\textbf{data} (,) a b = (,) a b}
\end{framed}
\caption{Тип-произведение}
\label{list:prod}
\end{figure}

Более сложным примером является рекурсивный тип бинарного дерева
\texttts{Tree}, определяемый c двумя конструкторами: \texttts{Leaf} для
листа и \texttts{Node} для узла, содержащего два корневых узла его
поддеревьев (листинг~\ref{list:bin-tree}). Тип \texttts{Tree}
представляет собой тип"/сумму единичного типа и типа"/произведения
двух типов"/констант.
\begin{figure}[h]
\begin{framed}
{\ttfamily\small
\textbf{data} Tree = Leaf | Node Tree Tree}
\end{framed}
\caption{Рекурсивный тип бинарного дерева}
\label{list:bin-tree}
\end{figure}

Вернёмся к примеру с определением обобщённой функции \texttts{eq}
(см.~листинг~\ref{list:geq}), рассмотренному в начале раздела. Мы
теперь можем, используя структуру алгебраических типов данных,
определить обобщённое представление для любого типа, являющегося
АТД. Определим, например, экземпляр класса \texttts{Generic}
(см.~листинг~\ref{list:generic}) для типа
\lstinline{Bool}~---~типа"/суммы двух единичных типов.

Для того, чтобы построить обобщённое представление,
соответствующее структуре типа \lstinline{Bool}, нам понадобится
ввести два типа"/комбинатора для суммы и единицы
(листинг~\ref{list:sum-unit}).
\begin{figure}[h]
\begin{framed}
\vspace{-0.25cm}
\begin{lstlisting}
data U         = Unit
data (a :+: b) = L a | R b
\end{lstlisting}
\vspace{-0.25cm}
\end{framed}
\caption{Комбинаторы для единичного типа и типа"/суммы}
\label{list:sum-unit}
\end{figure}

Определение типа обобщённого представления \texttts{Rep
\textbf{Bool}} с функциями \texttts{from} и \texttts{to} через введённые
комбинаторы выглядит, как в листинге~\ref{list:generic-bool}.
\begin{figure}[h]
\begin{framed}
\ttfamily\small
\textbf{instance} Generic \textbf{Bool} \textbf{where}\\
\ind\textbf{type} Rep \textbf{Bool} = U :+: U\\
\ind from True~~~~ = L Unit\\
\ind from False~~~ = R Unit\\
\ind to~~ (L Unit) = True\\
\ind to~~ (R Unit) = False
\end{framed}
\caption{Определение обобщённого представления для типа \lstinline{Bool}}
\label{list:generic-bool}
\end{figure}

Теперь можно определить экземпляры класса \texttts{GEq} из
листинга~\ref{list:geq} для типов"/комбинаторов
(листинг~\ref{list:geq-sum-unit}). В итоге функция \texttts{eq} будет
работать для типа \lstinline{Bool} и для любых алгебраических типов,
составленных из сумм и единиц и являющихся экземплярами класса
\texttts{Generic}.
\begin{figure}[h]
\begin{framed}
\ttfamily\small
\textbf{instance} (GEq a, GEq b) => GEq (a :+: b) \textbf{where}\\
\ind geq (L x) (L y) = geq x y\\
\ind geq (R x) (R y) = geq x y\\
\ind geq \_~~~~ \_~~~~ = False\\
\textbf{instance} GEq U \textbf{where}\\
\ind geq Unit Unit = True
\end{framed}
\caption{Определение работы \texttts{geq} для типов-сумм и единичных типов}
\label{list:geq-sum-unit}
\end{figure}

\subsection{Функторы}

Под термином \emph{функтор} в данной работе мы будем понимать
отображение, определённое на типах. Заметим, что это соответствует
теоретико"/категорному понятию функтора как отображения между
категориями, если рассматривать типы как категории, где объектами
типа \texttts{A} являются все возможные значения данного типа, а
морфизмами функции типа \texttts{A~"/>~A}. Формальное
определение функтора в теории категорий также требует выполнения
двух уравнений~---~сохранения единичного морфизма и композиции.
Однако мы в рамках этой работы будем пользоваться нестрогим
определением функтора и не требуем соблюдения этих свойств.

С программистской точки зрения, функтор~---~это полиморфный тип,
то есть тип, параметризованный другим типом. В языке Haskell введён
дополнительный уровень абстракции над типами~---~виды, или сорта,
типов (\textsl{kinds})~\cite{Loh2015}. Вид \texttts{*} соответствует
всем типам, значениями которых являются термы. Полностью
применённая форма любого типа (если он параметризован),
определённого через ключевое слово \lstinline{data}, имеет вид
\texttts{*}. Если \texttts{k} и \texttts{l}~---~виды, то типы вида
\texttts{l}, параметризованные типами вида \texttts{k}, будут иметь вид
\texttts{k~"/>~l}.

Например, типы \lstinline{Int} и список \lstinline{[Int]}~---~вида
\texttts{*}, а тип \texttts{[]}~---~неприменённая форма
параметризованного типа списка, представленного в
листинге~\ref{list:list},~---~вида \texttts{*~"/>~*}.
\begin{figure}[h]
\begin{framed}
\lstinline{data [] a = [] | a : [a]}
\end{framed}
\caption{Полиморфный тип списка}
\label{list:list}
\end{figure}

Функторы имеют вид \texttts{*~"/>~*}. Тип списка является примером
функтора.

Для того, чтобы строить обобщённые представления любых
функторов, необходимо ввести новую систему комбинаторов, как в
листинге~\ref{list:combinators}. Типы \texttts{K a}, \texttts{U},
\texttts{f~:+:~g} и \texttts{f~:\tms:~g} соответствуют
типу"/константе, единичному типу и типам суммы и произведения
(см.~листинги~\ref{list:base}--\ref{list:prod}), все они теперь
дополнительно параметризованы типом \texttts{x} и имеют вид
\texttts{*~"/>~*}. Тип \texttts{I} позволяет обобщённо представить
параметр функтора.
\begin{figure}[h]
\begin{framed}
\ttfamily\small
\textbf{data} K a~~~~~~ x = K a\\
\textbf{data} I~~~~~~~~ x = I x\\
\textbf{data} U~~~~~~~~ x = Unit\\
\textbf{data} (f :+: g) x = L (f x) | R (g x)\\
\textbf{data} (f :\tms: g) x = f x :\tms: g x
\end{framed}
\caption{Типы-комбинаторы для обобщённого представления функторов}
\label{list:combinators}
\end{figure}

С помощью новых комбинаторов можно представлять не только
функторы, но и вообще любые алгебраические типы данных. Такой
подход к построению обобщённого представления структуры типов
используется в~\cite{MuRec2009}.

Для определения обобщённого представления функторов требуется
также новый класс \texttts{Generic1} (листинг~\ref{list:generic1}), в
котором тип \texttts{Rep~f} будет вида \texttts{*~"/>~*}. В
листинге~\ref{list:generic-pair} приводится в качестве примера
представление функтора \lstinline{Pair Int}.
\begin{figure}[h]
\begin{framed}
\vspace{-0.25cm}
\begin{lstlisting}
class Generic1 f where
  type Rep1 f :: * -> *
  from1 :: f p -> Rep1 f p
  to1   :: Rep1 f p -> f p
\end{lstlisting}
\vspace{-0.25cm}
\end{framed}
\caption{Класс обобщённо представимых функторов}
\label{list:generic1}
\end{figure}
\begin{figure}[h]
\begin{framed}
\ttfamily\small
\textbf{data} Pair a b = Pair a b\\
\\
\textbf{instance} Generic1 (Pair Int) \textbf{where}\\
\ind\textbf{type} Rep1 (Pair Int) = K Int :\tms: I\\
\ind ...
\end{framed}
\caption{Пример типа обобщённого представления функтора}
\label{list:generic-pair}
\end{figure}

Необходимость введения отдельного класса для представления
функторов видится существенным недостатком, однако при
рассмотренном подходе для этого не существует другой возможности.
Такой способ принят в старой технологии обобщённого
программирования \textsf{GHC generics}~\cite{ghc-generics},
реализованной в компиляторе \textsf{GHC}. В
разделе~\ref{sec:generics-sop} будет показано, что новый подход
избавлен от этого недостатка.

\section{Зипперы}
\label{sec:zippers}

\section{Обобщённое программирование с \textsf{generics-sop}}
\label{sec:generics-sop}

\section{Реализация обобщённого зиппера}
\label{sec:generic-zippers-sop}

% Печать списка литературы (библиографии)
\printbibliography[heading=bibintoc%
    %,title=Библиография % если хочется это слово
]
% Файл со списком литературы: biblio.bib
% Подробно по оформлению библиографии:
% см. документацию к пакету biblatex-gost
% http://ctan.mirrorcatalogs.com/macros/latex/exptl/biblatex-contrib/biblatex-gost/doc/biblatex-gost.pdf
% и огромное количество примеров там же:
% http://mirror.macomnet.net/pub/CTAN/macros/latex/contrib/biblatex-contrib/biblatex-gost/doc/biblatex-gost-examples.pdf

\appendix
\ifthenelse{\value{worktype} > 1}{%
  \addtocontents{toc}{%
      \protect\renewcommand{\protect\cftchappresnum}{\appendixname\space}%
      \protect\addtolength{\protect\cftchapnumwidth}{\widthof{\appendixname\space{}} - \widthof{Глава }}%
  }%
}{
  \addtocontents{toc}{%
      \protect\renewcommand{\protect\cftsecpresnum}{\appendixname\space}%
      \protect\addtolength{\protect\cftsecnumwidth}{\widthof{\appendixname\space{}}}%
  }%
}

\section{Пример работы программы}

Здесь длинный листинг с примером работы.

\end{document}
